\section{Introduction} \label{sec:intro}

% ==================================================
% resumen
%% LO QUE VAMOS A CONTAR EN ESTE PAPER ES LO SIGUIENTE:
%primero mostramos algunos atractores en flotante
%para uno de esos a tractores mostramos los dominios de atracci—n
%
%segundo contamos como estudiamos la degradaci—n de los dominios de atracci—on.
%tercero contamos como estudiamos la degradaci—n de las variables de estado del sistema, es decir de la serie temporal generada por el sistema
%cuarto contamos como implementamos
%mostramos los resultados concretos obtenidos para la degradaci—n de los dominios de atracci—n y para la degradaci—n de las variables de estado
% ==================================================
%

Chaotic systems have an increasing number of applications and their implementation is specially involved due to the \textsl{extreme sensitivity to initial conditions} that present. In general, 
these systems are used for the generation of controlled noises, these digital pseudo-random noise generators (PRNGs) 
can be employed in a large number of electronic applications, such as encryption sequences for privacy, multiplexing techniques, 
electromagnetic compatibility and so on \cite{DeMicco2007A,DeMicco2007C,DeMicco2007B}. In computers and digital devices only \textsl{pseudo chaotic} attractors may be generated. But discretization may even destroy the \textsl{pseudo chaotic} behavior and consequently is a non trivial process. 

Among many chaotic systems available in the literature, we are interested in a family of $2D$-maps \cite{Sprott1993} proposed by Sprott,  and modelled by a pair of coupled quadratic equations:
%
\begin{equation}\label{eq:mapaSprott}
 \left\{\begin{aligned}
        x_{n+1}&=a_1~+~a_2~ x_n~+~a_3 ~x_n^2~+~a_4 ~x_n~ y_n~+~a_5 ~y_n~+~a_6~ y_n^2\\
        y_{n+1}&=a_7~+~a_8~ x_n~+~a_9 ~x_n^2~+~a_{10} ~x_n~ y_n~+~a_{11} ~y_n~+~a_{12}~ y_n^2
       \end{aligned}
 \right.
\end{equation}
%
where $\{x,y\}$ are the state variables and $\{a_i, i=1,\dots,12\}$
are the parameters. The reasons to study this particular system are two-fold: 

\begin{enumerate}
\item using floating-point arithmetic Sprott saw
that by automatic swept of parameters $a_i$ a huge number of
points in the parameter's space (about $6  .  10^{16}$) having
a chaotic permanent regime may be detected. He also
found a correlation between the correlation dimension and the
Lyapunov exponents of these chaotic attractors, with their
\textsl{visual appeal}, an interesting issue for automatic
\textsl{art} generation
\item it is possible to generate a novel encryption system because
they present multiple chaotic attractors depending on the selected
point in the parameter's space, either by replacing the S-box in AES \cite{Ahmad2013,Hussain2013}, or even by developing new encryption algorithms.
\end{enumerate}

Digital hardware implementation of dynamical systems, requires the use of
a finite number of bits to represent the state variables. Only rational numbers may be represented in a computer, in spite of the arithmetics used (fixed-point or floating-point arithmetics). From an engineering point of view, fixed-point arithmetic is more efficient than floating-point because it uses less resources, and each operation requires a lower number of clock cycles. As a consequence, power consumption is also diminished using fixed-point arithmetic. Floating-point architecture, on the other hand,  allows one to \textsl{recreate} the ideal system's trajectories in $\mathbb{R}^n$.

Only results for the analytical representation of the
maps in Eq. \ref{eq:mapaSprott} have been published in the open literature. The objective of
this paper is to extend the analysis to the digital version, to make
possible  the hardware implementation in fixed-point arithmetic.

Several strategies have been proposed in the literature for a correct
selection of the optimal number of bits in hardware
implementations. However, most of these procedures are limited to linear systems
\cite{Constantinides2002,Constantinides2003}. In digital
chaotic systems, a completely different behavior may be obtained by
varying the precision.  This issue  has gained interest recently,
and several new schemes have been proposed
\cite{Ding2007,Asseri2002,Azzaz2009}.

Grebogi's work \cite{Grebogi1988} showed that the average
length $T$  of periodic orbits of a dynamical system implemented in a computer, scales as a
function of the computer precision $\xi$  and the correlation dimension
 of the chaotic attractor, as $T \sim  \xi^{-d/2}$. In
\cite{SHUJUN2005} some findings on a new series of dynamical
indicators, which can quantitatively reflect the degradation
effects on a digital chaotic map realized with a fixed-point
finite precision have been reported, but they are restricted to $1D$
piecewise linear chaotic maps (PWLCM). In \cite{Dias2011} the effect of numerical precision on the mean
distance and on the mean coalescence time between trajectories of deterministic maps with either multiplicative noise parameter or with an additive noise term was investigated.

In this work we developed a detailed analysis of the
\textsl{degradation} of the multiatractor chaotic system modelled
by Eqs. \ref{eq:mapaSprott} as a fixed-point implementation is used. By
\textsl{degradation} we mean:  (a) the appearance of stable iced
points and stable periodic orbits with short periods, inside a
floating-point domain of attraction without stable orbits; (b) the
attractor itself becomes periodic and its statistical
characteristics change, making the system more deterministic. The main contributions of this paper are: 
\begin{itemize}
\item the analysis of the domains of
attraction of the chaotic attractors for a given set of parameters
as the number of bits (that encode the fractional part of the number)
increases; the appearance of stable fixed points and periodic
orbits with short periods are specially considered;
\item the  determination of the consequent \textsl{threshold width} for the bus, in order to make the  statistical
 properties of the digital implementation close to those of the floating-point implementation; 
\item  two different probability distribution functions (PDF) are assigned  to evaluate the stochasticity of the time series for different bus widths. Each PDF  $P$ is measured by the respective normalized Shannon entropy $H(P)$. These entropies have abrupt changes at specific bus widths. Period's lengths and \textsl{MLE} are also evaluated and results are compared with $H$s.
\end{itemize}
This work is organized as follows: first, section \ref{sec:estudio} comments some preliminary concepts and a few remarks on the problem that concern us. Section \ref{sec:chaos} gives a
brief description of the chaotic maps analyzed. Section \ref{sec:quanti} describes the quantifiers and the method used to study the degradation of the attractors. Section \ref{sec:repre} describes our proposed method in detail and emulate fixed-point representation. Then we give experimental results
in Section \ref{sec:results}. Finally, the conclusions are given in section \ref{sec:conclusions}.
