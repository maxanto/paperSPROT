\section{Conclusion} \label{sec:conclusions}
In this work, we have developed a detailed analysis of the changes in behaviour of a $2D$-quadratic map fixed-point implementation. Results show that compared to floating-point, fixed-point arithmetic executed on an integer datapath has a limited impact on the attraction domain and, also, in the characteristics of the sequences generated by the digitalized maps. We have found a threshold for the required 
number of bits where the system keeps the properties of the original (real) one. Our goal is to report the rate of degradation for each property, so as to be used by authors at the time of designing their particular applications. 
This is interesting because in many applications these maps are intended to be used as controlled noise generators and this system, in particular, admits the development of a novel encryption system.