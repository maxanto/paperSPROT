\section{Problem statement} \label{sec:estudio}

When iterating chaotic maps in $\mathbb{R}^2$, after a transient that depends on the mixing parameter ($r_{mix}$), the generated sequence limits in a point or a collection of points called attractor. A chaotic map can have one or more attractors. Attractor domain is called to all the initial conditions (ICs) that converge to each attractor. The ergodic sequences of the attractos, generated by the map, have a determined distribution called Invariant Probability Density Function (IPDF). Main characteristics of chaotic maps, IPDF and $r_{mix}$, can be obtained by calculating the Frobenius-Perron operator (FPO) which depends on the map's structure. The fixed points of its spectrum are the invariant densities and they correspond to the eigenvectors with eigenvalue equal to one, the mixing constant corresponds to the second largest eigenvalue of the FPO, \cite{Lasota1994,Lasota1973}.

When using finite precision, this analysis is not valid, all attractors take the form of fixed points or periodic orbits. The FPO of the map no longer describes the sequences' characteristics. Regarding the attractor domain, it will also change when digitalized, each initial value will be part of, or will converge to, a certain fixed point or periodic orbit. Generally, many new periodic orbits appear, and change when the number of bits employed varies.

With the adequate precision, periodic orbits of really extended periods can be reached. With the purpose of utilizing them in electronic applications it becomes necessary to understand how the attraction domain evolves with the variation of bits employed. It is mainly important to know which seeds, i. e. ICs, generate random-like outputs of the system, and also their period's lengths ($T$). Particular attention should be given to the \textsl{randomness degree} of the sequences, for this reason, some quantifiers were used here.

Using $n$ bits to represent the state variables of a $D$-dimensional system the maximum theoretical period $T_{max}$ that can be
reached is $T_{max}=2^{D.n}$. But some periodic orbits with period much lower than $T_{max}$, which are unstable in a floating-point arithmetic, become stable in fixed-point arithmetic, and viceversa. In principle, the modifications appear to be unpredictable.
The appearance of these low period stable orbits represents a \textsl{degradation} of the domains of attraction in the sense that certain initial conditions do not evolve toward the pseudo chaotic attractor. Then, to assure the desired pseudo chaotic behavior a threshold in $n_{min}$  exists. Consequently the hardware implementation requires the design of a
bus with at least this number of bits $n_{min}$. 
In this paper we want to emulate the behavior of a digital hardware implementation, making mandatory to 
exactly replicate the operation of the device. Our interest is
to measure how the domains of attraction degrade with a change in
the number of bits $n$ employed, as well as to find the threshold value 
$n_{min}$. 
